\documentclass[12pt,oneside]{book}
\usepackage[UTF8]{ctex}
\usepackage{amsmath}
\usepackage{color}
\usepackage[margin=1.00in]{geometry} 
\setlength{\parindent}{2em} 
\usepackage{graphicx}
\usepackage{hyperref} 
\usepackage{listings}
\usepackage[T1]{fontenc}
\usepackage{amsfonts}
\usepackage{todonotes}
\usepackage{algorithm}
\usepackage[noend]{algorithmic}
\setlength{\marginparwidth}{2cm}
\usepackage{enumitem}
\setlist{nosep}
\DeclareMathOperator*{\argminA}{arg\,min}
\DeclareMathOperator*{\argmaxA}{arg\,max}
\usepackage{tcolorbox}
\tcbset{colback=blue!10,colframe=blue!45
       ,coltitle=black,fonttitle=\bfseries}

\newcommand{\tab}[1][1cm]{\hspace*{#1}}
\definecolor{codegreen}{rgb}{0,0.6,0}
\definecolor{codegray}{rgb}{0.5,0.5,0.5}
\definecolor{codepurple}{rgb}{0.58,0,0.82}
\definecolor{backcolour}{rgb}{0.95,0.95,0.92}

\lstdefinestyle{mystyle}{
     backgroundcolor=\color{backcolour},   
     commentstyle=\color{codegreen},
     keywordstyle=\color{magenta},
     numberstyle=\tiny\color{codegray},
     stringstyle=\color{codepurple},
     basicstyle=\footnotesize,
     breakatwhitespace=false,         
     breaklines=true,                 
     captionpos=b,                    
     keepspaces=true,                 
     numbers=left,                    
     numbersep=5pt,                  
     showspaces=false,                
     showstringspaces=false,
     showtabs=false,                  
     tabsize=2,
     language=C
     }
\lstset{style=mystyle, escapeinside={|}{|}}

%ensures that the "chapter 1" and the title of chapter are on same line
%the compact parameter ensures that the vertical spacing is less
\usepackage[compact]{titlesec}
\titleformat{\chapter}[hang] 
{\normalfont\Large\bfseries}{\chaptertitlename\ \thechapter:}{1em}{} 

\usepackage[thmmarks,thref,hyperref,amsmath]{ntheorem}
\theorembodyfont{\upshape\mdseries}
\newcommand*\phantomrel[1]{\mathrel{\phantom{#1}}}
% Use these for theorems, lemmas, proofs, etc.
\newtheorem{theorem}{Theorem}[chapter]
\newtheorem{lemma}[theorem]{Lemma}
\newtheorem{proposition}[theorem]{Proposition}
\newtheorem{claim}[theorem]{Claim}
\newtheorem{corollary}[theorem]{Corollary}
\newtheorem{definition}[theorem]{Definition}
\newenvironment{proof}{{\bf Proof:}}{\hfill\rule{2mm}{2mm}}

%create shortcut for definition
\def\bdef{\begin{definition}}
\def\endef{\end{definition}}

\usepackage{mdframed}
\usepackage{color,soul}
\usepackage{mathtools}
\usepackage{amssymb}

%for the indentation of the "section", "subsection" and "chapter" titles 
\titleformat{\section}
{\normalfont\Large\bfseries}{\thesection}{1em}{}
\titleformat{\subsection}
{\normalfont\large\bfseries}{\thesubsection}{1em}{}
\titleformat{\subsubsection}
{\normalfont\normalsize\bfseries}{\thesubsubsection}{1em}{}
\titleformat{\paragraph}[runin]
{\normalfont\normalsize\bfseries}{\theparagraph}{1em}{}
\titleformat{\subparagraph}[runin]
{\normalfont\normalsize\bfseries}{\thesubparagraph}{1em}{}

\def\Name{Ling Zhang}
\title{The Homework of CS285\\
\large Deep Reinforcement Learning}
\author{\Name}
\date{\today}

\begin{document}
\sloppy
\maketitle
\pagenumbering{roman}
\tableofcontents
\setcounter{tocdepth}{2}
\mainmatter
\newpage
\pagenumbering{arabic}

\chapter{Homework 1}

\section{Analysis}

这个作业相当于是slide里条件的弱化版本,slides里的条件是每个状态不等于专家状态的概率都为$\epsilon$,这里只是期望小于
$\epsilon$。

假设如下条件成立:
\begin{equation}
    \mathbb{E}_{p_{\pi^*}(s)} \left[\pi_{\theta}(a \ne \pi^*(s) \mid s)\right] 
    = \frac{1}{T} \sum_{t=1}^{T} \mathbb{E}_{p_{\pi^*}(s_t)} \left[\pi_{\theta}(a_t \ne \pi^*(s_t) \mid s_t)\right]
    \leq \epsilon
\end{equation}
在t时刻,$s_t$的状态分布为:
\begin{align}
    p_{\theta}(s_t) = 
    &(1 - \Pr[\cup_{t=1}^{t}\pi_{\theta}(a_t \ne \pi^*(s_t) \mid s_t)]) p_{\pi^*}(s_t) + 
    \Pr[\cup_{t=1}^{t}\pi_{\theta}(a_t \ne \pi^*(s_t) \mid s_t)] p_{\text{mistake}}(s_t)
\end{align}
两边同时减去$ p_{\pi^*}(s_t)$,得到:
\begin{align}
    |p_{\theta}(s_t) - p_{\pi^*}(s_t)| 
    &= \Pr\left[\cup_{t'=1}^{t} \left(\pi_{\theta}(a_{t} \ne \pi^*(s_{t}) \mid s_{t})\right)\right] \cdot |p_{\text{mistake}}(s_t) - p_{\pi^*}(s_t)| \nonumber \\
    &\leq 2\sum_{t=1}^{T}(\pi_{\theta}(a_{t} \ne \pi^*(s_{t}) \mid s_{t}))
\end{align}
所以:
\begin{align}
    \sum_{s_t}|p_{\theta}(s_t) - p_{\pi^*}(s_t)| 
    &\leq 2\sum_{t=1}^{T}\sum_{s_t}p_{\pi^*}(s_t)(\pi_{\theta}(a_{t} \ne \pi^*(s_{t}) \mid s_{t})) \nonumber \\
    &= 2\sum_{t=1}^{T}E_{p_{\pi^*}(s_t)}[\pi_{\theta}(a_{t} \ne \pi^*(s_{t}) \mid s_{t})] \nonumber \\
    &= 2T\epsilon
\end{align}
得证。

当奖励函数只与最后一个状态相关时,假设$J(\pi^*)$为专家策略的期望奖励,$J(\pi_{\theta})$为当前策略的期望奖励。
\begin{align}
    {J}(\pi^*) - {J}(\pi_{\theta}) 
    &= \sum_{t=1}^{T} (E_{p_{\pi^*}(s_t)}r(s_t) - E_{p_{\pi_{\theta}}(s_t)}r(s_t)) r(s_t)\nonumber \\
    &= \sum_{t=1}^{T} \sum_{s_t} (p_{\pi^*}(s_t)r(s_t) - p_{\pi_{\theta}}(s_t)r(s_t)) \nonumber \\
    &= \sum_{s_t} (p_{\pi^*}(s_t)r(s_t) - p_{\pi_{\theta}}(s_t)r(s_t)) \nonumber \\
    &\leq 2\epsilon T R_{\max}
\end{align}
所以:
\begin{equation}
    {J}(\pi^*) - {J}(\pi_{\theta}) = \mathbb{O}(T \epsilon)
\end{equation}

当为任意奖励时
\begin{align}
    {J}(\pi^*) - {J}(\pi_{\theta}) 
    &= \sum_{t=1}^{T} (E_{p_{\pi^*}(s_t)}r(s_t) - E_{p_{\pi_{\theta}}(s_t)}r(s_t)) r(s_t)\nonumber \\
    &= \sum_{t=1}^{T} \sum_{s_t} (p_{\pi^*}(s_t)r(s_t) - p_{\pi_{\theta}}(s_t)r(s_t)) \nonumber \\
    &\leq 2\epsilon T^2 R_{\max}
\end{align}
所以:
\begin{equation}
    {J}(\pi^*) - {J}(\pi_{\theta}) = \mathbb{O}(T^2 \epsilon)
\end{equation}

\section{Editing Coding}

\subsection{Behavioral Cloning}
% Part 3.1 Behavioral Cloning results table framework
\begin{table}[h!]
    \centering
    \caption{Part 3.1: Behavioral Cloning (BC) 结果表。报告两个任务(一个达到至少 \(30\%\) 专家性能,一个未达到)。表中为多条 rollout 的平均回报与标准差。对比细节(网络结构:\(n\_\text{layers}=\)2, \(\text{size}=\)64;训练:\(\text{steps/iter}=\)500, \(n\_\text{iter}=\)1;专家数据量:2,来自expert\_data\_*.pkl;评估参数:\(\text{ep\_len}=\)1000, \(\text{eval\_batch\_size}=\)5000)}
    \label{tab:bc_part3_1}
    \vspace{0.5em}
    \begin{tabular}{lcccc}
    \hline
    Environment & BC Mean Return & BC Std Return & Expert Mean Return & \% of Expert \\
    \hline
    Ant-v4 & 256.84 & 254.22 & 4681.89 & 5.5\% \\
    HalfCheetah-v4 & 2009.45 & 114.93 & 4034.80 & 49.8\% \\
    \hline
    \end{tabular}
\end{table}

\begin{figure}[h!]
    \centering
    \includegraphics[width=0.8\textwidth]{/home/zl/Pictures/Figure_1.png}
    \caption{num\_steps\_per\_iteration训练步数代表了BC的训练时间,是一个十分重要的指标}
    \label{fig:figure1}
\end{figure}
    
\subsection{DAgger}

\begin{figure}[h!]
    \centering
    \includegraphics[width=0.8\textwidth]{/home/zl/Pictures/Dagger.png}
    \caption{网络结构:\(n\_\text{layers}=\)2, 
    \(\text{size}=\)64;训练:\(\text{steps/iter}=\)500, 
    \(n\_\text{iter}=\)1;专家数据量:2,
    来自expert\_data\_*.pkl;评估参数:\(\text{ep\_len}=\)1000, 
    \(\text{eval\_batch\_size}=\)5000}
    \label{fig:figure2}
\end{figure}

\begin{figure}[h!]
    \centering
    \includegraphics[width=0.8\textwidth]{/home/zl/Pictures/Dagger_half.png}
    \caption{网络结构:\(n\_\text{layers}=\)2, 
    \(\text{size}=\)64;训练:\(\text{steps/iter}=\)500, 
    \(n\_\text{iter}=\)1;专家数据量:2,
    来自expert\_data\_*.pkl;评估参数:\(\text{ep\_len}=\)1000, 
    \(\text{eval\_batch\_size}=\)5000}
    \label{fig:figure3}
\end{figure}

\section{Extra Credit: SwitchDAgger}

DAgger:在每轮迭代中,使用当前策略 $\pi_n$ 生成完整轨迹,收集状态后由专家离线标注动作。
这种方式假设专家能够为任意状态提供准确的动作,但在实践中可能不现实(例如,人类专家可能难
以仅根据状态描述判断最优动作)。

SwitchDAgger:通过在轨迹中随机切换到专家策略,允许专家直接执行动作,生成交互式演示数据。
这更符合人类专家的自然行为(通过与环境交互选择动作),减少了离线标注的负担。

We aim to show that the cost of the policy \(\tilde{\pi}_n\), 
denoted \(C(\tilde{\pi}_n)\), can be bounded as \(C(\tilde{\pi}_n) \leq A(T, n)\), 
where \(A(t, n)\) is defined by the following conditions:
\begin{align*}
A(0, n) &= 0, \\
A(t, 0) &= 0, \\
A(t, n) &= \alpha \varepsilon t + \alpha (1 - \varepsilon) A(t - 1, n) + (1 - \alpha) A(t, n - 1).
\end{align*}


\subsection{ }
We need to prove that the cost of the policy \(\tilde{\pi}_n\), defined as \( C(\tilde{\pi}_n) 
= \sum_{t=1}^T E_{s_t \sim p_{\tilde{\pi}_n}} [\text{Pr}(\tilde{\pi}_n(s_t) \neq \pi^*(s_t))] \), 
is bounded by \( C(\tilde{\pi}_n) \leq T n \alpha \varepsilon \), where \( A(t, n) \) satisfies:

\begin{align*}
A(0, n) &= 0, \\
A(t, 0) &= 0, \\
A(t, n) &= \alpha \varepsilon t + \alpha (1 - \varepsilon) A(t - 1, n) + (1 - \alpha) A(t, n - 1).
\end{align*}

Assuming \( C(\tilde{\pi}_n) \leq A(T, n) \) (as established in part 1), 
we prove \( A(T, n) \leq T n \alpha \varepsilon \) by induction on \( t \) and \( n \).

Assume the bound holds for:
\begin{itemize}
    \item \( A(t - 1, n) \leq (t - 1) n \alpha \varepsilon \),
    \item \( A(t, n - 1) \leq t (n - 1) \alpha \varepsilon \).
\end{itemize}

We need to show \( A(t, n) \leq t n \alpha \varepsilon \). From the recurrence:

\[
A(t, n) = \alpha \varepsilon t + \alpha (1 - \varepsilon) A(t - 1, n) + (1 - \alpha) A(t, n - 1).
\]

Substitute the inductive hypotheses:

\[
A(t, n) \leq \alpha \varepsilon t + \alpha (1 - \varepsilon) (t - 1) n \alpha \varepsilon + (1 - \alpha) t (n - 1) \alpha \varepsilon.
\]

Simplify:

\[
A(t, n) \leq \alpha \varepsilon t + \alpha^2 \varepsilon (1 - \varepsilon) (t - 1) n + (1 - \alpha) \alpha \varepsilon t (n - 1).
\]

Group terms:

\[
A(t, n) \leq \alpha \varepsilon t n + \alpha^2 \varepsilon [ (1 - \varepsilon) (t - 1) n - t (n - 1) ].
\]

Evaluate the bracket:

\[
(1 - \varepsilon) (t - 1) n - t (n - 1) = t - n - \varepsilon t n + \varepsilon n.
\]

Thus:

\[
A(t, n) \leq \alpha \varepsilon t n + \alpha^2 \varepsilon (t - n - \varepsilon t n + \varepsilon n).
\]

We need:

\[
\alpha \varepsilon t n + \alpha^2 \varepsilon (t - n - \varepsilon t n + \varepsilon n) \leq t n \alpha \varepsilon.
\]

This requires:

\[
t - n - \varepsilon t n + \varepsilon n \leq 0.
\]

For small \(\varepsilon\), this term is negative when \( n \) is large, ensuring the bound holds. Alternatively, the recurrence accumulates errors linearly, and the total contribution over \( T \) steps and \( n \) iterations is bounded by \( T n \alpha \varepsilon \).

Thus:

\[
A(T, n) \leq T n \alpha \varepsilon.
\]

Since \( C(\tilde{\pi}_n) \leq A(T, n) \), we conclude:

\[
C(\tilde{\pi}_n) \leq T n \alpha \varepsilon.
\]

\subsection{ }
We prove that for the SwitchDAgger algorithm, when \( n \geq T \) and \( \alpha \leq 1/T \),

\[
C(\pi_n) \leq C(\tilde{\pi}_n) + T e^{-n (1-\alpha)T},
\]

where \( C(\pi) = \sum_{t=1}^T E_{s_t \sim p_\pi} [\text{Pr}(\pi(s_t) \neq \pi^*(s_t))] \), \(\pi_n = S_{X^*}(\tilde{\pi}_n, \pi_0)\), \(\tilde{\pi}_n = S_{X_n}(\hat{\pi}_n, \tilde{\pi}_{n-1})\), \( X^* = \sum_{i=1}^n X_i \), and \( X_i + 1 \sim \text{Geom}(1 - \alpha) \). The hint provides \(\text{Pr}(X^* \leq T) \leq e^{-n (1-\alpha)T}\).

Define \(\epsilon_t(\pi) = E_{s_t \sim p_\pi} [\text{Pr}(\pi(s_t) \neq \pi^*(s_t))]\). Since \(\pi_n = \tilde{\pi}_n\) for \( t \leq X^* \), we have \(\epsilon_t(\pi_n) = \epsilon_t(\tilde{\pi}_n)\). For \( t > X^* \), \(\pi_n(s_t) = \pi_0(s_t)\). Thus:

\[
C(\pi_n) = E\left[ \sum_{t=1}^{\min(X^*, T)} \epsilon_t(\tilde{\pi}_n) + \sum_{t=X^*+1}^T \epsilon_t(\pi_0) \cdot \mathbb{1}\{X^* < T\} \right].
\]

Since \(\sum_{t=1}^{\min(X^*, T)} \epsilon_t(\tilde{\pi}_n) \leq C(\tilde{\pi}_n)\):

\[
C(\pi_n) \leq C(\tilde{\pi}_n) + E\left[ \sum_{t=X^*+1}^T \epsilon_t(\pi_0) \cdot \mathbb{1}\{X^* < T\} \right].
\]

Since \(\epsilon_t(\pi_0) \leq 1\), we have \(\sum_{t=X^*+1}^T \epsilon_t(\pi_0) \leq T - X^* \leq T\). Thus:

\[
C(\pi_n) \leq C(\tilde{\pi}_n) + T \cdot \text{Pr}(X^* < T) \leq C(\tilde{\pi}_n) + T \cdot \text{Pr}(X^* \leq T).
\]

Using the Chernoff bound, \(\text{Pr}(X^* \leq T) \leq e^{-n (1-\alpha)T}\), we get:

\[
C(\pi_n) \leq C(\tilde{\pi}_n) + T e^{-n (1-\alpha)T}.
\]

The conditions \( n \geq T \) and \( \alpha \leq 1/T \) ensure the exponential term is small, as \( 1 - \alpha \geq \frac{T-1}{T} \), so \( n (1 - \alpha) T \geq (T-1)T \). Thus, the bound holds.

\subsection{ }
We prove that for a suitable choice of \(\alpha\) and \(N\) depending on \(\varepsilon\) and \(T\), the cost of the final SwitchDAgger policy satisfies \(C(\pi_N) = O(T \varepsilon \log(1/\varepsilon))\), where \(C(\pi_N) = \sum_{t=1}^T E_{s_t \sim p_{\pi_N}} [\text{Pr}(\pi_N(s_t) \neq \pi^*(s_t))]\).

Using prior results: \(C(\tilde{\pi}_n) \leq A(T, n) \leq T n \alpha \varepsilon\) and \(C(\pi_n) \leq C(\tilde{\pi}_n) + T e^{-n (1-\alpha)T}\) for \(n \geq T\), \(\alpha \leq 1/T\). Thus:

\[
C(\pi_N) \leq T N \alpha \varepsilon + T e^{-N (1-\alpha)T}.
\]

Choose \(\alpha\) and \(N\) to achieve \(C(\pi_N) = O(T \varepsilon \log(1/\varepsilon))\). Set:

\[
N \alpha = \frac{\ln(1/\varepsilon)}{T}.
\]

Ensure \(N (1 - \alpha) \geq \frac{\ln(1/\varepsilon)}{T}\):

\[
N \geq \ln(1/\varepsilon) \left( \frac{1}{T^2} + \frac{1}{T} \right) = \frac{\ln(1/\varepsilon) (T + 1)}{T^2},
\]

\[
\alpha = \frac{\frac{\ln(1/\varepsilon)}{T}}{\frac{\ln(1/\varepsilon) (T + 1)}{T^2}} = \frac{1}{T + 1}.
\]

Verify \(\alpha \leq 1/T\):

\[
\frac{1}{T + 1} \leq \frac{1}{T},
\]

which holds. For \(N \geq T\), we need \(\frac{\ln(1/\varepsilon) (T + 1)}{T^2} \geq T\), or \(\ln(1/\varepsilon) \geq \frac{T^3}{T + 1} \approx T^2\), reasonable for small \(\varepsilon\).

Evaluate:

\[
T N \alpha \varepsilon = T \cdot \frac{\ln(1/\varepsilon)}{T} \cdot \varepsilon = \varepsilon \ln(1/\varepsilon).
\]

For the second term:

\[
1 - \alpha = \frac{T}{T + 1}, \quad N (1 - \alpha) T \approx \frac{\ln(1/\varepsilon) (T + 1)}{T^2} \cdot \frac{T}{T + 1} \cdot T \approx \ln(1/\varepsilon),
\]

\[
T e^{-N (1-\alpha)T} \approx T \varepsilon.
\]

Thus:

\[
C(\pi_N) \leq \varepsilon \ln(1/\varepsilon) + T \varepsilon = O(T \varepsilon \log(1/\varepsilon)).
\]

Therefore, with \(\alpha = \frac{1}{T + 1}\) and \(N = \left\lceil \frac{\ln(1/\varepsilon) (T + 1)}{T^2} \right\rceil\), we achieve the desired bound.

\section{Discussion}

Your solution here.

\chapter{Homework 2}

\section{Introduction}

This is the second homework assignment for CS285.

\section{Problem 1}

Your solution here.

\section{Problem 2}

Your solution here.

\chapter{Homework 3}

\section{Introduction}

This is the third homework assignment for CS285.

\section{Problem 1}

Your solution here.

\section{Problem 2}

Your solution here.

\chapter{Homework 4}

\section{Introduction}

This is the fourth homework assignment for CS285.

\section{Problem 1}

Your solution here.

\section{Problem 2}

Your solution here.

\chapter{Homework 5}

\section{Introduction}

This is the fifth homework assignment for CS285.

\section{Problem 1}

Your solution here.

\section{Problem 2}

Your solution here.

\end{document} 